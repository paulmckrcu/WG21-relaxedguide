\documentclass{article}

% standard packages

% A more pleasant font
\usepackage[T1]{fontenc} % use postscript type 1 fonts
\usepackage{textcomp} % use symbols in TS1 encoding
\usepackage{mathptmx,helvet,courier} % use nice, standard fonts for roman, sans and monospace respectively

% Improves the text layout
\usepackage{microtype}

\usepackage{lscape}
\usepackage{fancyhdr}
\usepackage{epsfig}
\usepackage{subfigure}
\usepackage{url}
\usepackage{graphics}
\usepackage{enumerate}
\usepackage{ifthen}
\usepackage{float}
\usepackage{listings}
\lstset{basicstyle=\ttfamily}
% \usepackage[strings]{underscore}
% \usepackage{underscore}
\usepackage[bookmarks=true,bookmarksnumbered=true,pdfborder={0 0 0}]{hyperref}

\lstset{
  literate={\_}{}{0\discretionary{\_}{}{\_}}%
}

\DeclareUrlCommand\email{}

\pagestyle{fancy}
\rhead{}

\newfloat{listing}{tbp}{lol}
\floatname{listing}{Listing}

\begin{document}
\title{A Relaxed Guide to \co{memory_order_relaxed}}

\newcommand{\co}[1]{\lstinline[breaklines=yes,breakatwhitespace=yes]{#1}}

\author{
Hans Boehm\\\email{hboehm@google.com} \and
Paul E.~McKenney\\\email{paulmck@kernel.org} \and
The Indefatigible TBD
}
\date{August 9, 2019}
\maketitle{}

\begin{abstract}
	The out-of-thin-air (OOTA) and read-from-untaken-branch (RFUB)
	properties of the specification of \co{memory_order_relaxed}
	have resulted in considerable consternation over the years.
	Although there are no known instances of full-blown OOTA
	behavior, and no known RFUB-induced failures of production code,
	the theoretical possibility of these properties severely
	complicate automated analysis of large C and C++ code bases.
	Thus far, attempts to eliminate OOTA and RFUB properties from
	the memory model have resulted in otherwise needless added
	overheads on weakly ordered systems on the one hand or
	excessive implementation complexity on the other.
	However, \co{memory_order_relaxed} never was intended to be used
	in arbitrary code, but rather as a part of deliberate application
	of specific concurrency designs.
	This paper forms an initial catalog of patterns underlying such
	designs.
\end{abstract}

\section{OOTA and RFUB Background}
\label{sec:OOTA and RFUB Background}

@@@

Loads of citations and summaries.

Heuristics, such as single-direction data flow.

\section{Relaxed Design Patterns}
\label{sec:Relaxed Design Patterns}

Many taken from Hans's \co{memory-model-design} posting on
September 4, 2018.\footnote{
	Message-ID: \co{<CAMOCf+jchGw6DeE2NyCJA3wfFbNH-WFn59JruZPSWt9_jPW9NQ@mail.gmail.com>}.}

\subsection{Untrusted Loads}
\label{sec:Untrusted Loads}

In many cases, it is acceptable for a load from an atomic shared variable
to occasionsally return random bits because the value is checked by
some later operation.
In such cases, the load can be a relaxed load.

Perhaps the most well-known later checking operation is compare-and-swap (CAS).
The \co{std::atomic_compare_exchange_*()} family of read-modify-write
CAS operations are typically used in a loop, and often require an initial
load prior to the first pass through that loop.
For non-relaxed CAS operations, this initial load can typically be a
relaxed load, with the CAS operation's ordering preventing OOTA and RFUB
behaviors.
Relaxed CAS operations need to be part of some other design pattern if they
are to be OOTA/RFUB-free in conjunction with an initial relaxed load.
One common design pattern is the single-location data structure discussed in
Section~\ref{sec:Single-Location Data Structures}.

Additional examples are presented by
Sinclair et al.~\cite{Sinclair:2017:CAR:3079856.3080206}.

\subsection{Owner Field for Re-Entrant Mutex}
\label{sec:Owner Field for Re-Entrant Mutex}

\emph{Note:} There is some debate on this design pattern.

A re-entrant exclusive mutex must track its owner in order to avoid
self-deadlock when the owner re-acquires the mutex it already holds.
This owner field is updated only while the lock is held, and its value
is used only to compare for equality to the current thread's ID.
Other threads can access the owner concurrently with the owner's
update, so the owner field must be an atomic in order to avoid data races.

However, the only time that the owner field can be equal to the thread ID
is when that thread carried out the last update to the owner field and
still holds the mutex.
No other thread is allowed to write to the owner field while the lock
is held, so it is impossible to form the cycles required for OOTA or
RFUB behavior to manifest.\footnote{
	In fact, the accesses to the owner field and the mutex's acquire
	and release operations will be sequentially consistent.}
Therefore, both the reads from and the writes to the owner field
may use \co{memory_order_relaxed}.

More generally, if an object is written only while a given mutex is
held, all accesses to that object may be relaxed without the possibility
of OOTA or RFUB behaviors.

Even more generally, if an object must be atomic for other reasons, then
any non-racing read or write can be a relaxed access.\footnote{
	This covers case~\#8 in Hans's September 4, 2019 email.}

\subsection{Statistical Counters}
\label{sec:Statistical Counters}

Statistical counters have concurrent updates and reads, and thus must
use atomics.
However, the concurrent reads are modeled as returning approximate results
(for example, for monitoring or debugging),
and can in fact be modeled as sequentially consistent approximate
operations.
Furthermore, data flow is almost always unidirectional, proceeding from
the updater responding to an event and flowing to the reader, precluding
the cycles required for OOTA or RFUB behavior to manifest.

Exact values are sometimes obtained from statistical counters in
stop-the-world situations, such as checking for consistent results
at the end of a stress test or benchmarking run.
In these cases, OOTA and RFUB behaviors are avoided due to the
single-threaded nature of the consistency checks.

\subsection{Single-Location Data Structures}
\label{sec:Single-Location Data Structures}

Relaxed atomic operations provide sequentially consistent access to
a single object.
This means that data structures that fit into a single object can
be accessed with relaxed atomics with no possibility of OOTA or
RFUB behavior.

Note well that a pair of single-location data structures might well
interact in a way that could raise the spectre of OOTA or RFUB.

\subsection{Single-Threaded Processing Phases}
\label{sec:Single-Threaded Processing Phases}

Relaxed atomic operations also provide sequentially consistent access
to multiple objects from a single thread.
This allows relaxed atomic operations to be used during the single-threaded
initialization and cleanup phases of a concurrent program.

\subsection{Shared Fences}
\label{sec:Shared Fences}

The \co{std::atomic_thread_fence()} function can be used to order
multiple sets of accesses, for example, by replacing a series of
acquire loads with relaxed loads followed by an
\co{std::atomic_thread_fence(memory_order_acquire)}~\cite[Section 4.1]{RaulSilvera2007WeakMemoryModel}
or a series of release stores with an
\co{std::atomic_thread_fence(memory_order_release)} followed by
a series of relaxed stores~\cite[Section 4.2]{RaulSilvera2007WeakMemoryModel}.

In this design pattern, OOTA and RFUB behaviors are ruled out by the semantics
of \co{std::atomic_thread_fence()}.

\subsection{One-Way Memory Allocation}
\label{sec:One-Way Memory Allocation}

``For allocating fresh memory that's never deallocated, or deallocated
with a heavy weight one-sided mechanism.''

\emph{[ How do we characterize this to avoid the infamous RFUB example?
	Here is my first cut: ]}

Note that memory allocation and pointer publication are separate
design patterns, and that publication of a pointer to other threads
requires at least \co{memory_order_relaxed} in order to ensure that
the allocation happens before the publication, even if the allocated
memory remains uninitialized.

\emph{[ @@@ Add citation for RFUB example. ]}

\subsection{Sequence Locking}
\label{sec:Sequence Locking}

The accesses within a sequence-locking read-side critical section can
used relaxed loads because any concurrency with the corresponding
update will result in a retry, thus discarding the loaded values.
This prevents an OOTA or RFUB cycle.

\subsection{Java Hashcode Access}
\label{sec:Java Hashcode Access}

\emph{[ Does this apply to C or C++? ]}

\subsection{Atomic Reference-Count Updates}
\label{sec:Atomic Reference-Count Updates}

In certain reference-count use cases, the ordering of the incremeents and
decrements is irrelevant.
One common case is where it is only legal to increment the reference
count when you already have a reference, in which case the count cannot
possibly decrease to zero in the meantime.
Because only the one-to-zero transition requires ordering, reference-count
increments can be relaxed in cases where another reference is guaranteed
to be held throughout.

\subsection{Relaxed Consumption}
\label{sec:Relaxed Consumption}

In cases where a full-speed \co{memory_order_consume} is needed on a
weak-memory system and where the developers are willing to live within
strict coding standards~\cite{PaulEMcKenney2014rcu-dereference},
\co{memory_order_relaxed} may be used to head dependency chains.

Note well that this design pattern is outside of the current standard.

\subsection{Chaotic Relaxation}
\label{sec:Chaotic Relaxation}

There are a number of iterative numerical algorithms for which unsynchronized
access does not slow convergence as much as waits for barrier synchronization.
These algorithms can use relaxed loads and stores to update the numerical
data~\cite{Andrews91textbook}.

In theory, these algorithms are subject to OOTA and RFUB behaviors, however,
in practice, current implementations avoid such behaviors.

\section{Marking of Relaxed Design Patterns}
\label{sec:Marking of Relaxed Design Patterns}

It is currently believed that these design patterns must be explicitly marked
in order for code reviewers and automatic verifiers to recognize them
and validate their usage.
Here are some candidate marking strategies that have been discussed
within the C++ standards committee:

\begin{enumerate}
\item	Create new \co{memory_order} \co{enum} members for each new
	design pattern.
	This has the benefit of calling out the pattern in an unmistakable
	way that is visible to the compiler, but requires that each
	new design pattern be standardized.
	It also does not support the case where a given access plays a
	role in multiple overlapping design patterns.
\item	Use structured comments to mark each design pattern.
	This avoids the time delays and administrative overhead inherent
	in standardization, and could potentially allow multiple comments
	to handle a given access that plays a role in multiple overlapping
	design patterns.
\item	Define C++ \co{template} types for each design pattern.
	This is an excellent idea where it applies, as it might well
	for the statistical counters discussed in
	Section~\ref{sec:Statistical Counters}.
	However, we have reason to doubt that \co{template} types can
	be reasonably created for all possible relaxed-access design
	patterns.
\end{enumerate}

More ideation and discussion is needed on this topic.

\section{Use of Relaxed Design Patterns}
\label{sec:Use of Relaxed Design Patterns}

@@@

\section{Concluding Remarks}
\label{sec:Concluding Remarks}

@@@


\bibliographystyle{plain}
\bibliography{bib/RCU,bib/WFS,bib/hw,bib/os,bib/parallelsys,bib/patterns,bib/perfmeas,bib/refs,bib/syncrefs,bib/search,bib/swtools,bib/realtime,bib/TM,bib/standards}

\end{document}
